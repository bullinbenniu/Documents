% Created 2015-08-09 Sun 22:39
\documentclass[11pt]{article}
\usepackage[utf8]{inputenc}
\usepackage[T1]{fontenc}
\usepackage{fixltx2e}
\usepackage{graphicx}
\usepackage{longtable}
\usepackage{float}
\usepackage{wrapfig}
\usepackage{rotating}
\usepackage[normalem]{ulem}
\usepackage{amsmath}
\usepackage{textcomp}
\usepackage{marvosym}
\usepackage{wasysym}
\usepackage{amssymb}
\usepackage{hyperref}
\tolerance=1000
\usepackage[vcentermath]{youngtab}
\usepackage{braket}
\usepackage{mathrsfs}
\newcommand{\bm}[1]{\mbox{\boldmath{$#1$}}}
\author{CuriousBull}
\date{\today}
\title{RGM}
\hypersetup{
  pdfkeywords={},
  pdfsubject={},
  pdfcreator={Emacs 24.5.1 (Org mode 8.2.10)}}
\begin{document}

\maketitle
\tableofcontents

\section{RGM}
\label{sec-1}
\subsection{9-Quark System}
\label{sec-1-1}
\subsubsection{DineutronLambda System}
\label{sec-1-1-1}
\begin{enumerate}
\item Theoretic Concepts
\label{sec-1-1-1-1}
\begin{enumerate}
\item A 9-Quart System can be divided into three baryon subsystem.
\item Each baryon system will be treated locally.
\item The interaction between each cluster remains unknown.
\item Complete set, Gaussian Functions will be adopted to describe interaction funct-
ions.
\item To simplify our case, \textbf{color part} of wave function will be treated as three s-
inglet wave functions.
\item Fully antisymmetrization should be applied to wave functions.
\item Wave function of center-of-mass system is added to transform wave functions in-
to productions of single partical wave functions.
\end{enumerate}
\item Complex Analysis I (Single Baryon)
\label{sec-1-1-1-2}
\begin{enumerate}
\item Wave function of a single baryon can be written as:
\begin{equation}
\psi(123)=\phi^{spatial}(123)\chi^{color}(123)\eta^{\sigma\cdot{}flavor}(123)\\
\end{equation}
Among which,\\
\begin{eqnarray}
\phi(123)&=&\phi(1)\phi(2)\phi(3)\\
\phi(i)&=&\left(\frac{1}{\pi{}b^{2}}\right)^{\frac{3}{4}}\rm{exp}(-\frac{\bm{r}_{i}^2}{2b^2})\\
\chi(123)&=&\young(1,2,3)\quad\young(r,g,b)\\
\eta^{\sigma\cdot{}flavor}&=&\frac{1}{\sqrt{2}}\left(\young(12,3)\quad
\young(uu,d)\quad\young(12,3)\quad\young(\alpha\alpha,\beta)
+\young(13,2)\quad\young(uu,d)\quad\young(13,2)\quad\young(\alpha\alpha,\beta)\right)
\end{eqnarray}
\item Hamiltonian operator for single baryon can be written as:
\begin{equation}
H = \sum_{i}^{3}(m_i+\frac{\bm{p}^{2}_{i}}{2m_i})+\sum_{i>j=1}^{3}V_{i j}-T_{c.m}
\end{equation}
Among which,\\
\begin{eqnarray}
V_{i j} & = & V_{i j}^C + V_{i j}^G\\
V_{i j}^C & = & -a_c\bm{\lambda_i}\cdot\bm{\lambda_j}\cdot\bm{r}_{i j}\\
V_{i j}^G & = & \alpha_s\frac{\bm{\lambda_i}\cdot\bm{\lambda_j}}{4}\left[\frac{1}
{|\bm{r}_{i j}|}-\frac{\pi}{2}\delta(\bm{r}_{i j})\left(\frac{1}{m_{i}^2}+\frac{1}{m_{j}^2}
+ \frac{4\bm{\sigma}_i\cdot\bm{\sigma}_j}{3m_{i}m_{j}}\right)\right]\\
T_{c.m} & = & \frac{(\sum\limits_{i=1}^{3}\bm{p}_{i})^{2}}{2M_{total}}
\end{eqnarray}
\item To simple our calculation, Jacobi's transformation is applied:
\begin{eqnarray}
\frac{\bm{r}_1+\bm{r}_2}{2}&=&\bm{R}\\
\bm{r}_1-\bm{r}_2&=&\bm{r}_{1 2}\equiv\bm{r}
\end{eqnarray}
and then, we obtain,\\
\begin{eqnarray}
\bm{r}_{1}^{2}+\bm{r}_{2}^{2}&=&\frac{\bm{r}^2+4\bm{R}^2}{2}\\
{\rm d}\bm{r}_1{}{\rm d}\bm{r}_2&=&{\rm d}\bm{r}{\rm d}\bm{R}
\end{eqnarray}
\item Some matrix elements calculation steps are listed below:
\begin{eqnarray}
\bra{\phi(123)}\bm{r}_{1 2}\ket{\phi(123)}&=&\bra{\phi(1)\phi(2)}\bm{r}_{1 2}\ket{\phi(1)\phi(2)}\nonumber\\
& = & \left(\frac{1}{\pi{}b^2}\right)^3\iint{}\rm{exp}(-\frac{\bm{r}_{1}^2+\bm{r}_{2}^2}{2b^2})\bm{r}_{1 2}^2
\rm{exp}(-\frac{\bm{r}_{1}^2+\bm{r}_{2}^2}{2b^2}){\rm d}{\bm r}_1{}{\rm d}{\bm r}_2\nonumber\\
&=&\left(\frac{1}{\pi{}b^2}\right)^3\iint{}\rm{exp}(-\frac{\bm{r}^2+4\bm{R}^2}{2b^2})\bm{r}^2{\rm d}{\bm r}{\rm d}{\bm R}\nonumber\\
&=&\left(\frac{1}{\pi{}b^2}\right)^3\times{}4\pi\int_{0}^{\infty}\rm{exp}(-\frac{2{\bm{R}^2}}{b^2})\bm{R}^2{\rm d}{\bm R}\times{}
\nonumber\\
&&{}\times{}4\pi\int_{0}^{\infty}\rm{exp}(-\frac{\bm{r}^2}{2b^2}){\rm d}{\bm r}\nonumber\\
&=&\left(\frac{1}{\pi{}b^2}\right)^3\times{}4\pi\times\frac{1}{4}\frac{b^3\sqrt{\pi}}{2\sqrt{2}}\times{}4\pi\times
\frac{3b^5}{8}\times{}4\sqrt{2\pi}\nonumber\\
&=&3b^2
\end{eqnarray}
In same way, terms include $\frac{1}{\bm{r}}$ and $\delta(\bm{r})$ can be calculated easily. Results are listed below:
\begin{eqnarray}
\bra{\phi(123)}\frac{1}{|\bm{r}_{1 2}|}\ket{\phi(123)}&=&\bra{\phi(1)\phi(2)}\bm{r}_{1 2}\ket{\phi(1)\phi(2)}\nonumber\\
&=&\frac{1}{b}\sqrt{\frac{2}{\pi}}\\
\bra{\phi(123)}\delta(\bm{r}_{1 2})\ket{\phi(123)}&=&\bra{\phi(1)\phi(2)}\bm{r}_{1 2}\ket{\phi(1)\phi(2)}\nonumber\\
&=&\left(\frac{1}{2\pi{}b^2}\right)^{\frac{3}{2}}
\end{eqnarray}
To calculate terms include $\bm{p}_i$, we need to calculate $\bra{\phi(i)}\bm{p}_{i}^{2}\ket{phi(i)}$,
\begin{eqnarray}
\bra{\phi(123)}\bm{p}_{1}\ket{\phi(123)}&=&\bra{\phi(1)}\bm{p}_{1}\ket{\phi(1)}\nonumber\\
&=&\int_{0}^{\infty}\rm{exp}(-\frac{r_{1}}{2b^2})\left[-\frac{1}{r_{1}^2}\frac{\partial}{\partial r_1}(r_{1}^2\frac{\partial}{\partial r_1}
exp(-\frac{r_1}{2b^2}))\right]r_{1}^2\rm{d}r_{1}sin\theta\rm{d}\theta{}\rm{d}\phi\nonumber\\
&=&4\pi\int_{0}^{\infty}\rm{exp}(-\frac{r_{1}}{2b^2})(\frac{3}{b^2}-\frac{r_{1}^2}{b^4})r_{1}^2\rm{d}r_{1}\nonumber\\
&=&4\pi(\frac{3b}{4}\sqrt{\pi}-\frac{3b}{8}\sqrt{\pi})
\end{eqnarray}
For term like $\sum\limits_{i>j=1}^{3}\frac{\bm{\sigma}_i\cdot\bm{\sigma}_j}{m_i\cdot{}m_j}$, it's convenient to do some transformation:
\begin{enumerate}
\item $m_1=m_2=m_3$
\begin{eqnarray}
\sum\limits_{i>j=1}^{3}\frac{\bm{\sigma}_i\cdot\bm{\sigma}_j}{m_i\cdot{}m_j}&=&4\times\frac{1}{m_{1}^2}\left[\bm{S}_1\cdot\bm{S}_2+
\bm{S}_1\cdot\bm{S}_3+\bm{S}_2\cdot\bm{S}_3\right]\nonumber\\
&=&\frac{2}{m_{1}^2}\left[(\bm{S}_{1}+\bm{S}_{2}+\bm{S}_{3})^2-(\bm{S}_{1}^2+\bm{S}_{2}^2+\bm{S}_{3}^2)\right]\nonumber\\
&=&\frac{2}{m_{1}^2}\left[\bm{S}_{total}(\bm{S}_{total}+1)-\frac{9}{4}\right]
\end{eqnarray}
\item $m_1=m_2\neq{}m_3$
\begin{eqnarray}
\sum\limits_{i>j=1}^{3}\frac{\bm{\sigma}_i\cdot\bm{\sigma}_j}{m_i\cdot{}m_j}&=&4\left[\frac{\bm{S}_1\cdot\bm{S}_2}
{m_{1}^2}+\frac{1}{m_1\cdot{}m_3}
(\bm{S}_1+\bm{S}_2)\cdot\bm{S}_3\right]\nonumber\\
&=&\frac{2}{m_{1}^2}\left[(\bm{S}_{1}+\bm{S}_{2})^2-\bm{S}_{1}^2-\bm{S}_{2}^2\right]+\frac{2}{m_1{}m_3}
\left[\bm{S}_{total}^2-(\bm{S}_{1}+\bm{S}_{2})^2-\bm{S}_{3}^2\right]\nonumber\\
&=&\frac{2}{m_{1}^2}\left[\bm{S}_{\alpha}(\bm{S}_{\alpha}+1)-\frac{3}{2}\right]+\frac{2}{m_1{}m_3}\times{}
\nonumber\\
&&{}\times\left[\bm{S}_{total}(\bm{S}_{total}+1)-{}
\bm{S}_{\alpha}(\bm{S}_{\alpha}+1)-\frac{3}{4}\right]
\end{eqnarray}
\end{enumerate}
\end{enumerate}
\item Complex Analysis II (Few Body Problem)
\label{sec-1-1-1-3}
Here are some points need to be considered carefully:
\begin{enumerate}
\item To deal with matrix elements in a unified way, we transfer single body operator into a two-body operator.
Then matrix elements like below need to be considered,
\begin{equation}
\bra{\bm{x}_{1}^\prime\bm{x}_{2}^\prime}H_{i j}\ket{\bm{x}_{1}\bm{x}_{2}}
\end{equation}
A general expression for this can be transfered like:
\begin{eqnarray}
\bra{\bm{x}_{1}^\prime\bm{x}_{2}^\prime}H_{i j}\ket{\bm{x}_{1}\bm{x}_{2}}&=&
\left(\frac{1}{2b^2}\right)^{\frac{3}{2}}\rm{exp}(-\frac{1}{4b^2}\left[(\bm{x}_{1}^\prime-\bm{x}_{1})^2
+(\bm{x}_{2}^\prime-\bm{x}_{2})^2\right])\times{}\nonumber\\
&&{}\times{}\int\hat{H_{i j}}\rm{exp}(-\frac{(\bm{r}-\frac{1}{2}\bm{\rho})^2}{2b^2}\rm{d}\bm{r}
\end{eqnarray}
Among which,
\begin{eqnarray}
\bm{r}&=&\bm{r}_1-\bm{r}_2\\
\bm{\rho}=(\bm{x}_{1}+\bm{x}_{1}^\prime)-(\bm{x}_{2}+\bm{x}_{2}^\prime)
\end{eqnarray}
\item The handle of terms include $\frac{1}{m}$, we've used a special method.
For a case with 8 u quark and 1 s quark, the $\frac{1}{m}$ can be expressed as:
\begin{equation}
\frac{1}{m}=\frac{8}{9m_u}+\frac{1}{9m_s}
\end{equation}
\item Some calculations of terms are listed below:
For convenience, we denote,
\begin{equation}
\mathscr{O}_2=\rm{exp}(-\frac{1}{4b^2}\left[(\bm{x}_{1}^\prime-\bm{x}_{1})^2
+(\bm{x}_{2}^\prime-\bm{x}_{2})^2\right])
\end{equation}
then, we obtain terms,
\begin{eqnarray}
\bra{\bm{x}_{1}^\prime\bm{x}_{2}^\prime}m_{1}+m_{2}\ket{\bm{x}_{1}\bm{x}_{2}}&=&2m\mathscr{O}_2\\
\bra{\bm{x}_{1}^\prime\bm{x}_{2}^\prime}\frac{\bm{p}_{1}^2}{2m_{1}}+\frac{\bm{p}_{2}^2}{2m_{2}}\ket{\bm{x}_{1}\bm{x}_{2}}
&=&\frac{3\hbar^2}{4mb^2}\left[2-\frac{(\bm{x}_{1}^\prime-\bm{x}_{1})^2
+(\bm{x}_{2}^\prime-\bm{x}_{2})^2}{6b^2}\right]\mathscr{O}_2\nonumber\\
&=&\frac{\hbar^2}{8mb^4}\left[12b^2-(\bm{x}_{1}^\prime-\bm{x}_{1})^2
+(\bm{x}_{2}^\prime-\bm{x}_{2})^2\right]\mathscr{O}_2\\
\bra{\bm{x}_{1}^\prime\bm{x}_{2}^\prime}\bm{p}_1\cdot\bm{p}_2\ket{\bm{x}_{1}\bm{x}_{2}}&=&
-\frac{\hbar^2[(\bm{x}_{1}^\prime-\bm{x}_{1})^2
+(\bm{x}_{2}^\prime-\bm{x}_{2})^2]}{4b^4}\mathscr{O}_2\\
\bra{\bm{x}_{1}^\prime\bm{x}_{2}^\prime}\bm{r}^2\ket{\bm{x}_{1}\bm{x}_{2}}&=&(3b^2+\frac{\bm{\rho}^2}{4})\mathscr{O}_2\\
\bra{\bm{x}_{1}^\prime\bm{x}_{2}^\prime}\frac{1}{r}\ket{\bm{x}_{1}\bm{x}_{2}}&=&\frac{2}{\rho}erf(\frac{\rho}{\sqrt{8b^2}})\mathscr{O}_2\\
\bra{\bm{x}_{1}^\prime\bm{x}_{2}^\prime}\delta(r)\ket{\bm{x}_{1}\bm{x}_{2}}&=&\left(\frac{1}{2\pi{}b^2}\right)^{\frac{3}{2}}
\rm{exp}(-\frac{\bm{\rho}^2}{8b^2})\mathscr{O}_2
\end{eqnarray}
The specific calculation steps are listed below:
\begin{eqnarray}
\bra{\bm{x}_{1}^\prime\bm{x}_{2}^\prime}m_{1}+m_{2}\ket{\bm{x}_{1}\bm{x}_{2}}&=&\left(\frac{1}{2\pi{}b^2}\right)^{\frac{3}{2}}
\mathscr{O}_2\times{}2m\int\rm{exp}(-\frac{r^2+\frac{1}{4}\rho^2-r{}\rho{}cos\theta}{2b^2})\times{}\nonumber\\
&&{}\times{}r^2sin\theta\rm{d}r\rm{d}\theta\rm{d}\phi\nonumber\\
&=&\left(\frac{1}{2\pi{}b^2}\right)^{\frac{3}{2}}\mathscr{O}_2\times{}2m
\int\rm{exp}(-\frac{r^2+\frac{1}{4}\rho^2-r{}\rho{}cos\theta}{2b^2})\times{}\nonumber\\
&&{}\times{}r^2sin\theta\frac{\rm{d}\left(-\frac{r^2+\frac{1}{4}\rho^2-r{}\rho{}cos\theta}{2b^2}\right)}
{-\frac{r{}\rho{}sin\theta}{2b^2}}
\rm{d}r\rm{d}\phi\nonumber\\
&=&\left(\frac{1}{2\pi{}b^2}\right)^{\frac{3}{2}}\mathscr{O}_2\times{}2m(-\frac{2b^2}{\rho})\times2\pi\times{}\nonumber\\
&&{}\int(\rm{exp}(-\frac{r^2+\frac{1}{4}\rho^2+r{}\rho}{2b^2})-\rm{exp}
(-\frac{r^2+\frac{1}{4}\rho^2-r{}\rho}{2b^2}))r\rm{d}r\rm{d}\phi\nonumber\\
&=&\left(\frac{1}{2\pi{}b^2}\right)^{\frac{3}{2}}\mathscr{O}_2\times{}2m(-\frac{2b^2}{\rho})\times2\pi\times{}\nonumber\\
&&{}\int(\rm{exp}(-\frac{(r+\frac{1}{2}\rho)^2}{2b^2})(r+\frac{\rho}{2})-\rm{exp}
(-\frac{(r-\frac{1}{2}\rho)^2}{2b^2})(r-\frac{\rho}{2})-{}\nonumber\\
&&-{}\rm{exp}(-\frac{(r+\frac{1}{2}\rho)^2}{2b^2})(\frac{\rho}{2})-
\rm{exp}(-\frac{(r-\frac{1}{2}\rho)^2}{2b^2})(\frac{\rho}{2}))
r\rm{d}r\rm{d}\phi\nonumber\\
&=&\left(\frac{1}{2\pi{}b^2}\right)^{\frac{3}{2}}\mathscr{O}_2\times{}2m(-\frac{2b^2}{\rho})\times{}2\pi\times{}\nonumber\\
&&\left(-\rho\cdot\frac{\sqrt{\pi}}{2(\frac{1}{2b^2})^{\frac{1}{2}}}\right)\nonumber\\
&=&2m\mathscr{O}_2
\end{eqnarray}
\end{enumerate}
\end{enumerate}
% Emacs 24.5.1 (Org mode 8.2.10)
\end{document}